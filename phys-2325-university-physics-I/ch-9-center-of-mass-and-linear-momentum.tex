\section{Center of mass and linear momentum}
\subsection{Center of mass}
\noindent $\star$ The center of mass of a system of particles is the point that moves as though (1) all of the system’s mass were concentrated there and (2) all external forces were applied there.

The center of mass of a system of \( n \) particles is the weighted average of their positions:
\[
x_{\text{com}} = \frac{1}{M} \sum_{i=1}^{n} m_i x_i, \qquad
y_{\text{com}} = \frac{1}{M} \sum_{i=1}^{n} m_i y_i, \qquad
z_{\text{com}} = \frac{1}{M} \sum_{i=1}^{n} m_i z_i,
\]
or in vector form,
\[
\vec{r}_{\text{com}} = \frac{1}{M} \sum_{i=1}^{n} m_i \vec{r}_i,
\]
where \( M \) is the total mass of the system.

\subsection{Newton's second law for a system of particles}

The motion of a system’s center of mass follows Newton’s second law:
\[
\vec{F}_{\text{net}} = M \vec{a}_{\text{com}},
\]
where \( \vec{F}_{\text{net}} \) is the total external force, \( M \) is the total mass, and \( \vec{a}_{\text{com}} \) is the acceleration of the center of mass.

\subsection{Linear momentum}
\noindent $\star$ The time rate of change of the momentumof a particle is equal to the net force
acting on the particle and is in the direction of that force.

\noindent $\star$ The linear momentum of a system of particles is equal to the product of the total
mass M of the system and the velocity of the center of mass.

For a single particle, linear momentum is defined as
\[
\vec{p} = m\vec{v},
\]
a vector in the same direction as velocity.  
Newton’s second law can be written as
\[
\vec{F}_{\text{net}} = \frac{d\vec{p}}{dt}.
\]

For a system of particles,
\[
\vec{p} = M\vec{v}_{\text{com}}, \qquad \vec{F}_{\text{net}} = \frac{d\vec{p}}{dt}.
\]

\subsection{Collision and impulse}
Newton’s second law in momentum form gives the impulse–momentum theorem:
\[
\vec{p}_f - \vec{p}_i = \Delta \vec{p} = \vec{J},
\]
where impulse is
\[
\vec{J} = \int_{t_i}^{t_f} \vec{F}(t)\,dt.
\]
For constant average force,
\[
J = F_{\text{avg}} \Delta t.
\]

When multiple bodies of mass \( m \) collide steadily with a fixed object,
\[
F_{\text{avg}} = -\frac{n}{\Delta t} m \Delta v = -\frac{\Delta m}{\Delta t} \Delta v.
\]

\subsection{Conservation of linear momentum}
\noindent $\star$ If no net external force acts on a system of particles, the total linear momentum $\vec{P}$ of the system cannot change.

\noindent $\star$ If the component of the net external force on a closed system is zero along an
axis, then the component of the linear momentum of the system along that axis cannot change.


In a closed and isolated system (no external forces), the total momentum remains constant:
\[
\vec{P} = \text{constant}.
\]

This can also be written as
\[
\vec{P}_i = \vec{P}_f.
\]

\subsection{Momentum and kinetic energy in collisions}
In an inelastic collision, kinetic energy is not conserved, but total momentum is:
\[
\vec{p}_{1i} + \vec{p}_{2i} = \vec{p}_{1f} + \vec{p}_{2f}.
\]

For motion along one axis:
\[
m_1 v_{1i} + m_2 v_{2i} = m_1 v_{1f} + m_2 v_{2f}.
\]

If the bodies stick together, the collision is \textit{completely inelastic} and they share a common final velocity \(V\).  
The velocity of the system’s center of mass remains unchanged during the collision.

\subsection{Elastic collisions in one dimention}
\noindent $\star$ In an elastic collision, the kinetic energy of each colliding body may change, but
the total kinetic energy of the system does not change.

In an elastic collision, both kinetic energy and momentum are conserved.  
For a one-dimensional collision between two bodies (1 and 2), the final velocities are:

\[
v_{1f} = \frac{m_1 - m_2}{m_1 + m_2} v_{1i}, \qquad
v_{2f} = \frac{2m_1}{m_1 + m_2} v_{1i}.
\]

Elastic collisions conserve total energy and momentum within the system.

\subsection{Collisions in two dimentions}
If two bodies collide and their motion is not along a single axis, the collision is \textit{two-dimensional}.  
For a closed and isolated system, momentum is conserved in both directions:
\[
\vec{p}_{1i} + \vec{p}_{2i} = \vec{p}_{1f} + \vec{p}_{2f}.
\]

In component form, this gives two equations (for \(x\) and \(y\) directions).  
If the collision is also elastic, kinetic energy is conserved:
\[
K_{1i} + K_{2i} = K_{1f} + K_{2f}.
\]

\subsection{Systems with various mass: A rocket}
In the absence of external forces, a rocket’s acceleration is governed by the thrust equation:
\[
R v_{\text{rel}} = M a,
\]
where \(M\) is the rocket’s instantaneous mass, \(R\) is the fuel consumption rate, and \(v_{\text{rel}}\) is the exhaust speed of the fuel relative to the rocket.

For a rocket with constant \(R\) and \(v_{\text{rel}}\), whose mass changes from \(M_i\) to \(M_f\) while its velocity changes from \(v_i\) to \(v_f\), the velocity change is given by:
\[
v_f - v_i = v_{\text{rel}} \ln \frac{M_i}{M_f}.
\]
