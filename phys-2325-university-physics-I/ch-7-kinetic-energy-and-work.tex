\section{Kinetic energy and work}
\subsection{Kinetic Energy}
The kinetic energy $K$ associated with the motion of a particle of mass $m$ and speed $v$, where $v$ is well below the speed of light, is
\[
K = \tfrac{1}{2}mv^2 \quad \text{(kinetic energy).}
\]

\subsection{Work and kinetic energy}

$\star$ Work W is energy transferred to or from an object by means of a force acting on
the object. Energy transferred to the object is positive work, and energy transferred from the object is negative work.\\
Work $W$ is the energy transferred to or from an object by a force acting over a displacement $\vec{d}$. For a constant force,
\[
W = Fd\cos\phi = \vec{F} \cdot \vec{d},
\]
where $\phi$ is the angle between $\vec{F}$ and $\vec{d}$.
\\
$\star$ To calculate the work a force does on an object as the object movesthrough some
displacement, we use only the force component along the object’s displacement.
The force component perpendicularto the displacement does zero work.\\
$\star$ A force does positive work when it has a vector component in the same direction
as the displacement, and it does negative work when it has a vector component in
the opposite direction. It does zero work when it has no such vector component.\\

Only the component of $\vec{F}$ along $\vec{d}$ does work. The net work equals the change in kinetic energy:
\[
\Delta K = K_f - K_i = W,
\]
or equivalently,
\[
K_f = K_i + W.
\]

\subsection{Work done by the gravitational force}

The work done by the gravitational force on an object of mass $m$ over displacement $\vec{d}$ is
\[
W_g = mgd \cos \phi,
\]
where $\phi$ is the angle between $\vec{F}_g$ and $\vec{d}$.

The work done by an applied force $W_a$ and gravity are related by
\[
\Delta K = K_f - K_i = W_a + W_g.
\]
If $K_f = K_i$, then
\[
W_a = -W_g,
\]
meaning the applied force does as much work on the object as gravity removes.

\subsection{Work done by a spring force}

The force from a spring is given by Hooke's law:
\[
\vec{F}_s = -k\vec{d}, \quad \text{or} \quad F_x = -kx,
\]
where $k$ is the spring constant and $\vec{d}$ (or $x$) is the displacement from equilibrium.

The work done by a spring when moving from $x_i$ to $x_f$ is
\[
W_s = \tfrac{1}{2}k x_i^2 - \tfrac{1}{2}k x_f^2.
\]
$\star$ Work $W_s$
is positive if the block ends up closer to the relaxed position (x = 0) than
it was initially.It is negative if the block ends up farther away from x = 0.It is zero
if the block ends up at the same distance from x = 0.\\

If $x_i = 0$ and $x_f = x$, then
\[
W_s = -\tfrac{1}{2}kx^2.
\]
$\star$ If a block that is attached to a spring isstationary before and after a displacement,then the work done on it by the applied force displacing it is the negative
of the work done on it by the spring force. \\

\subsection{Work done by a general variable force}

If the force $\vec{F}$ on an object depends on position, the total work is found by integrating the force over the path:
\[
W = \int_{x_i}^{x_f} F_x\,dx + \int_{y_i}^{y_f} F_y\,dy + \int_{z_i}^{z_f} F_z\,dz.
\]
If the force acts only along the $x$-axis, this simplifies to
\[
W = \int_{x_i}^{x_f} F(x)\,dx.
\]
\\
\subsection{Power}

Power is the rate at which a force does work.

The \textit{average power} over a time interval $\Delta t$ is
\[
P_{\text{avg}} = \frac{W}{\Delta t}.
\]

The \textit{instantaneous power} is
\[
P = \frac{dW}{dt}.
\]

For a force $\vec{F}$ making an angle $\phi$ with velocity $\vec{v}$,
\[
P = Fv \cos \phi = \vec{F} \cdot \vec{v}.
\]

