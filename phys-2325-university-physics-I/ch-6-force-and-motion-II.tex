\section{Force and motion - II}

\subsection{Friction}
Two types of friction:
\begin{itemize}
    \item Static frictional force - Prevents motion when a body is at rest. Is written as $\vec{f}_s$.
    \item Kinetic frictional force - What friction becomes when sliding begins. Is written as $\vec{f}_k$.
\end{itemize}
Properties of friction:
\begin{enumerate}
    \item If a body does not move, the static frictional force $\vec{f}_s$ balances the component of the applied force that is parallel to the surface.

    \item The maximum static friction is given by
    \[
    f_{s,\text{max}} = \mu_s F_N,
    \]
    where $\mu_s$ is the coefficient of static friction and $F_N$ is the normal force.

    \item Once the body starts sliding, the frictional force becomes kinetic and is given by
    \[
    f_k = \mu_k F_N,
    \]
    where $\mu_k$ is the coefficient of kinetic friction.
\end{enumerate}

\subsection{The drag force and terminal speed}
When a body moves through a fluid, it experiences a drag force $\vec{D}$ opposing its motion. The drag magnitude depends on the relative speed $v$ and is given by
\[
D = \tfrac{1}{2} C \rho A v^2,
\]
where $C$ is the drag coefficient, $\rho$ is the fluid density, and $A$ is the cross-sectional area.

At terminal velocity, the drag force equals the gravitational force, and the speed becomes constant:
\[
v_t = \sqrt{\frac{2F_g}{C \rho A}}.
\]

\subsection{Uniform Circular Motion}
$\star$ A centripetal force accelerates a body by changing the direction of the body’s
velocity without changing the body’s speed.\\
A particle moving in a circle of radius $R$ at constant speed $v$ has a centripetal acceleration directed toward the center:
\[
a = \frac{v^2}{R}.
\]
This acceleration is caused by a net centripetal force of magnitude
\[
F = \frac{mv^2}{R},
\]
where $m$ is the particle's mass.
