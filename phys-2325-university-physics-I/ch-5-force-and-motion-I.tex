\section{Force and motion - I}

\subsection{Newton's first and second laws}
$\star$ \textbf{Newton's First Law:} If no \textit{net} force acts on a body ($\vec{F}_{\text{net}} = 0$), the body's velocity cannot change; that is, the body cannot accelerate.\\\\
$\star$ An inertial reference frame is one in which Newton’s laws hold.\\\\
$\star$ \textbf{Newton's Second Law:} The net force on a body is equal to the product of the body's mass and its acceleration.
\begin{itemize}
    \item The law can be written as:
    \[
    \vec{F}_{\text{net}} = m\vec{a} \quad \text{(Newton's second law).}
    \]
    \item In component form:
    \[
    F_{\text{net},x} = ma_x, \qquad
    F_{\text{net},y} = ma_y, \qquad
    F_{\text{net},z} = ma_z.
    \]
\end{itemize}
$\star$ The acceleration component along a given axis is caused \textit{only by} the sum of the force components along that \textit{same} axis, and not by force components along any other axis.

\subsection{Some particular forces}
\[
F_g = mg.
\]
$\star$ The weight $W$ of a body is equal to the magnitude $F_g$ of the gravitational force on the body.
\[
W = mg \quad \text{(weight)},
\]
$\star$ When a body presses against a surface, the surface (even a seemingly rigid one) deforms and pushes on the body with a normal force $\vec{F}_N$ that is perpendicular to the surface.

\subsection{Applying Newton's laws}
$\star$ \textbf{Newton's Third Law:} When two bodies interact, the forces on the bodies from each other are always equal in magnitude and opposite in direction.
\[
F_{BC} = F_{CB} \quad \text{(equal magnitudes)}
\]