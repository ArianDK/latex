\documentclass[fleqn]{article}
\usepackage{amsmath}
\usepackage{multicol}

\usepackage[top=0.7in, bottom=0.7in, left=0.7in, right=0.7in]{geometry}

\title{Fundamentals of Physics}
\author{Arian DK}
\date{\today}

\begin{document}
\setcounter{section}{2}

\maketitle

\section{Vectors}

\subsection{Vectors and their components}
\begin{itemize}
    \item \textbf{Vector} - has magnitude and direction.
    \item \textbf{Scalar} - quantities that can are fully described by a magnitude (a numerical value alone), without any direction.
    \item \textbf{Vector sum (resoultant)} - are the product from adding two or more vecotrs.
\end{itemize}
\begin{align*}
\vec{s} &= \vec{a} + \vec{b}, \\[6pt]
\vec{a} + \vec{b} &= \vec{b} + \vec{a} \quad &\text{(commutative law)} \\[6pt]
(\vec{a} + \vec{b}) + \vec{c} &= \vec{a} + (\vec{b} + \vec{c}) \quad &\text{(associative law)} \\[6pt]
\vec{d} &= \vec{a} - \vec{b} = \vec{a} + (-\vec{b}) \quad &\text{(vector subtraction)}
\end{align*}
A component of a vector is the projection of a vector on an axis.
\begin{center}
Finding the components:\\
$a_x = a\cos\theta 
\quad \text{and} \quad 
a_y = a\sin\theta$
\end{center}
If we know a vectors $a_x$ and $a_y$ and want magnitude or angle we can use:
\begin{center}
$a = \sqrt{a^2_x + a^2_y} \quad \text{and} \quad \theta = \tan^{-1}(\frac{a_y}{a_x})$
\end{center}

\subsection{Unit vectors, adding vectors by components}
\textbf{Unit vector} - is a vector with magnitude of exactly 1.
\begin{align*}
      & r_x = a_x + b_x \\
\vec{r} = \vec{a} + \vec{b} \quad & r_y = a_y + b_y \\
      & r_z = a_z + b_z
\end{align*}
We can write a vector $\vec{a}$ in terms of unit vectors as: $\vec{a} = a_x \hat{i} + a_y \hat{j} + a_z \hat{k}$

\newpage
\subsection{Multiplying vectors}
There are two ways of multiplying vectors, one way produces a scalar (scalar product) and the other way produces a new vector (vector product):
\begin{table}[h]
    \centering
    \begin{tabular}{|c|c|c|}
        \hline
        Feature & Scalar product (dot) & Vector product (cross) \\ 
        \hline
        Symbol & $\vec{A} \cdot \vec{B}$ & $\vec{A} \times \vec{B}$ \\ \hline
        Result & Scalar (number) & Vector \\ \hline
        Formula & $AB\cos\theta$ & $AB\sin\theta$ \\ \hline
        Component form & 
        $A_xB_x + A_yB_y + A_zB_z$ & 
        $(A_yB_z - A_zB_y,\; A_zB_x - A_xB_z,\; A_xB_y - A_yB_x)$ \\ \hline
    \end{tabular}
\end{table}




\end{document}
